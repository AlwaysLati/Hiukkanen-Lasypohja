\song{Korttipakka}{Tapio Rautavaara}

Tämä on vanha tarina \\
Kerran sota-aikana tapahtui näin \\
Oli oltu pitkällä marssilla \\
ja komppania oli tullut pieneen kaupunkiin \\
Seuraava päivä oli sunnuntai \\
ja vääpeli komensi pojat kirkkoon \\
Kun pappi luki rukouksen, niin ne, \\
joilla oli rukouskirja, ottivat sen esiin \\

\hspace{10cm} \\

Mutta yhdellä sotilaalla ei ollut muuta kuin korttipakka, \\
jonka hän levitti eteensä kirkon penkille \\
Vääpeli huomasi kortit ja sanoi: \\
"Sotamies! Kortit pois!" \\
Kirkonmenojen jälkeen sotamies vangittiin \\
ja vietiin kenttäoikeuteen \\

\hspace{10cm} \\

Sotatuomari kysyi: "Perustelkaa tekonne, \\
muuten rankaisen teitä kovemmin kuin ketään koskaan!" \\
Sotamies vastasi: "Herra sotatuomari, \\
olen ollut marssilla melkein koko viikon \\
Eikä minulla ole Raamattua eikä rukouskirjaa \\
Mutta toivon, että ymmärrätte vilpittömän selvitykseni" \\
Näin sanoen sotamies aloitti kertomuksensa: \\

\hspace{10cm} \\

"Katsokaas, herra sotatuomari, \\
kun näen ässän, muistan, \\
että on vain yksi Jumala \\

\hspace{10cm} \\

Kakkosesta muistan, \\
että Raamatussa on kaksi osaa, \\
Vanha ja Uusi Testamentti \\

\hspace{10cm} \\

Kun näen kolmosen, \\
ajattelen Isää, Poikaa ja Pyhää Henkeä \\

\hspace{10cm} \\

Kun näen nelosen, \\
tulevat mieleeni neljä Evankelistaa: \\
Matteus, Markus, Luukas ja Johannes \\

\hspace{10cm} \\

Kun näen viitosen, \\
ajattelen viittä viisasta neitsyttä, \\
jotka laittoivat lamppunsa kuntoon \\
Yhteensä heitä oli kymmenen; \\
viisi ymmärtäväistä, jotka pelastuivat, \\
viisi tyhmää, joilta ovi suljettiin \\

\hspace{10cm} \\

Kun näen kuutosen, tulee mieleeni, \\
että kuutena päivänä Jumala loi taivaan ja maan \\

\hspace{10cm} \\

Kun näen seiskan, muistan, \\
että seitsemäntenä päivänä Jumala lepäsi \\

\hspace{10cm} \\

Ja kun näen kahdeksikon, \\
ajattelen niitä kahdeksaa oikeamielistä, \\
jotka Jumala pelasti vedenpaisumuksesta \\
Ne olivat Nooa, hänen vaimonsa, \\
heidän poikansa ja poikiensa vaimot \\

\hspace{10cm} \\

Kun näen yhdeksikön, \\
ajattelen spitaalisia, \\
jotka Vapahtajamme puhdisti vaivoistaan \\
Yhdeksän kymmenestä ei edes kiittänyt häntä siitä \\

\hspace{10cm} \\

Kun näen kympin, \\
ajattelen kymmentä käskyä, \\
jotka Jumala antoi Moosekselle kivitaulussa \\

\hspace{10cm} \\

Kun näen kuninkaan, ajattelen, \\
että on vain yksi Taivasten Valtakunnan kuningas, \\
Jumala kaikkivaltias \\

\hspace{10cm} \\

Ja kun näen kuningattaren, \\
ajattelen Neitsyt Maariaa, \\
joka on Taivasten Valtakunnan kuningatar \\

\hspace{10cm} \\

Sotilas on paholainen \\

\hspace{10cm} \\

Kun lasken yhteen korttipakan pisteet, \\
saan tulokseksi 364 \\
Niitä on vain yksi vähemmän \\
kuin on päiviä vuodessa \\

\hspace{10cm} \\

Kortteja on 52 \\
Yhtä monta kuin on viikkoja vuodessa \\

\hspace{10cm} \\

Maita on neljä, \\
yhtä monta kuin on viikkoja kuukaudessa \\

\hspace{10cm} \\

Kuvakortteja on yhteensä 12, \\
yhtä monta kuin on kuukausia vuodessa \\

\hspace{10cm} \\

Kussakin maassa on 13 korttia, \\
yhtä monta kuin on viikkoja neljännesvuodessa \\

\hspace{10cm} \\

Kuten huomaatte, herra sotatuomari, \\
korttipakka on minulle Raamattu, almanakka ja rukouskirja \\
Hyvät kuulijat, tämä tarina on tosi! \\