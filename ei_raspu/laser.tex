\song{Laser}{}

Ensimmäinen laser kehitettiin\\
yhdysvalloissa vuonna 60.\\
Laseria käytetään muun muassa\\
CD-levyjen lukemiseen.\\
Laseria käytetään teollisuudessa\\
hitsauksessa ja leikkauksessa.\\
Lääketieteessä laseria käytetään\\
kirurgisiin tarkoituksiin.\\
\hspace{10mm}\\
Maailman tehokkaimmat laserit\\
pystyvät noin petawatin hetkelliseen tehoon.\\
Vuonna 2009\\
yliopiston fyysikot kertoivat aikeesta\\
nostaa tehoja tuhatkertaisesti\\
yhteen eksawattiin.\\
\hspace{10mm}\\
Laserin vaarallisuus perustuu\\
pitkään kantamaan ilmassa.\\
Jo pienen tehon laserin säteily\\
voi aiheuttaa näkövaurioita.\\
Laserit on turvallisuusluokiteltu\\
luokkiin I–IV.\\
IV-laser voi polttaa\\
iholle palovamman hetkessä.\\
\hspace{10mm}\\
Maailman tehokkaimmat laserit…\\
\hspace{10mm}\\
Laser on monille tuttu tietokonepeleistä.\\
Laser on monille tuttu sci-fi elokuvista.\\
\hspace{10mm}\\
Erityisesti science fictionissa laser esitetään\\
usein virheellisesti.\\
Toisin kuin Star Wars -elokuvissa,\\
lasersäde ei koskaan näy tyhjiössä.\\
Lasersäteen eteneminen on yleensä esitetty\\
liian hitaana.\\
Muistuttaen perinteisen\\
suorasuuntausammuksen etenemistä\\
\hspace{10mm}\\
Maailman tehokkaimmat laserit…\\