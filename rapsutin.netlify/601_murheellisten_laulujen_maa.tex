\song{Murheellisten laulujen maa (\textit{601})}{}

        Syyttömänä syntymään sattui hän \\
        tähän maahan pohjoiseen ja kylmään, \\
        jossa jo esi-isät juovuksissa tottakai \\
        hakkasivat vaimot, lapset \\
        jos ne kiinni sai \\
\hspace{10mm} \\
        Perinteisen miehen kohtalon \\
        halus' välttää poika tuo \\
        En koskaan osta kirvestä, \\
        enkä koskaan viinaa juo \\
        muuten juon talon \\
\hspace{10mm} \\
        Lumihanki kutsuu perhettä talvisin, \\
        vaan en tahdo tehdä koskaan lailla isin \\
        Mut kun työnvälityksestä työtä ei saa, \\
        hälle kohtalon koura juottaa väkijuomaa \\
\hspace{10mm} \\
        Niin Turmiolan Tommi taas herää henkiin \\
        ja herrojen elkeet tarttuvat renkiin. \\
        Kohti laukkaa viinakauppaa. \\
\hspace{10mm} \\
        Se miehen epätoivoon ajaa, \\
        kun halla viljaa korjaa. \\
        Keskeltä kylmän mullan hiljaa \\
        kylmä silmä tuijottaa, \\
        kun kirves kohoaa. \\
        Keskeltä kumpujen, mullasta maan \\
        isät ylpeinä katsovat poikiaan. \\
\hspace{10mm} \\
        Työttömyys, viina, kirves ja perhe \\
        lumihanki, poliisi ja viimeinen erhe. \\
        Tämä tuhansien murheellisten laulujen maa \\
        jonka tuhansiin järviin juosta saa. \\
\hspace{10mm} \\
        Katajainen kansa, jonka itsesäälin määrää \\
        ei mittaa järki, \\
        eikä Kärki määrää, \\
        jonka lauluissa hukkuvat elämän valttikortit \\
        ja kiinni pysyvät taivahan portit. \\
        Einari Epätoivosta, \\
        ne kertovat. \\
