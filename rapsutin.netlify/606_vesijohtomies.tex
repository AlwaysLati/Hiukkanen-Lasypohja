\song{Vesijohtomies (\textit{606})}{Kerenski / Vaarilla on saari}

        Oli kerran aika, jolloin olin ma \\
        aika hauska poika kaikkien katsella. \\
        Tyttöjä mä riiasin, ryyppäsin ja laulelin. \\
        Olinhan mä ammatiltain vesijohtomies. \\
\hspace{10mm} \\
        Iltamista myöhään, kun me palattiin \\
        Kaisan asunnolle, niin siellä pantihin \\
        leikiks' kaikki entinen, sillä Kaisa tiesi sen, \\
        ett' olinhan mä ammatiltain vesijohtomies. \\
\hspace{10mm} \\
        Kaisa sitten haastoi minut käräjiin \\
        ja syyte oli siitä, ett' liiaks' halasin. \\
        Tuomar' multa kysäsi, ett' mikä on sun toimesi. \\
        Kaisa siihen vastas': Hän on vesijohtomies. \\
\hspace{10mm} \\
        Maailmalle sinne tiedon oivan sain, \\
        että pieni poika on nyt Kaisallain. \\
        Tuota vaan oon tuuminut, liekö Kaisa muistanut, \\
        ett' olenhan mä ammatiltain vesijohtomies. \\
\hspace{10mm} \\
        Onnenpäivät meiltä ei pääty ensinkään, \\
        yhdessä kun Kaisan kanssa niitä vietetään. \\
        Illalla kun nukutaan korvaani hän kuiskuttaa: \\
        Sinä olet minun oma vesijohtomies. \\
