\song{SIK-hymni (\textit{3})}{}

            Kun mä fuksina tänne taapersin, \\
            en tiennyt mitä sähkö on. \\
            Meni innosta pieni pääni sekaisin, \\
            laulukirja oli tarpeeton. \\
            Mulla kavaletti, nabla ja kosini fii \\
            oli öisinkin mielessä; \\
            hetken taistelin, itkin, sitten ymmärsin: \\
            tässä jotakin on pielessä. \\
\hspace{10mm} \\
            :,: Sillä arvonimi “teekkari” sovi sulle ei, \\
            jos vain Maxwell aikasi vei. \\
            Kylmää kaljaa juo ja nauti \\
            nuoruudesta. :,: \\
\hspace{10mm} \\
            \begin{comment} Lisäsäkeistöt: \\
\hspace{10mm} \\
            Dippainssinä sitten paljastan \\
            vihdoin kaiken, mitä sähkö on. \\
            vetyfuusion voiman kun mä valjastan, \\
            siintää elonpolku puutteeton. \\
            Mulla rommilevy, ällikorttikooditkin \\
            on jo kirkkaana mielessä. \\
            Monimeedian keksin, nettisysteemin \\
            - silti kuiva pinta kielessä. \\
\hspace{10mm} \\
            :,: Sillä arvonimi... \\
\hspace{10mm} \\
            Väikkärinkin jo kansiin saaneena \\
            yhä mietin, mitä sähkö on. \\
            Kirjaa vaalin mä suurena aarteena, \\
            muttei sekään ole aukoton. \\
            Vaikka proffana tieteitä opetan, \\
            pala kurkkuun nousee niellessä. \\
            Fuksin huomioon luentoni lopetan: \\
            - Tuoksut tiedontuskanhielle sä! \\
\hspace{10mm} \\
            :,: Sillä arvonimi... \\
\end{comment}