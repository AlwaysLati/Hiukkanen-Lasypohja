\song{Teemu Teekkari (\textit{1})}{}

1.1 Lyökäämme lasit pojat yhtehen \\
\hspace{10mm} \\
            Lyökäämme lasit pojat yhtehen, \\
            hurraa, hurraa hurrataan vaan, \\
            :,: ja me juodaan niin kauan, kuin me jaksetaan \\
            ja me onnelliset oomme vaan. :,: \\
\hspace{10mm} \\
            Minä vietin niin iloista elämää, \\
            hurraa, hurraa hurrataan vaan, \\
            :,: että heilanikin rupes mua epäilemään. \\
            Minä onnellinen olen vaan. :,: \\
\hspace{10mm} \\
            Läksin minä merille seilaamaan, \\
            hurraa, hurraa hurrataan vaan, \\
            :,: siniaalloilta onneani etsimään. \\
            Minä onnellinen olen vaan. :,: \\
\hspace{10mm} \\
            Sieltä löysinkin, mitä etsinkin, \\
            hurraa, hurraa hurrataan vaan, \\
            :,: sain tuon iloisen luontoni takaisin. \\
            Minä onnellinen olen vaan. :,: \\
\hspace{10mm} \\
            Ennen kuin minä lauluni lopetan, \\
            hurraa, hurraa hurrataan vaan, \\
            :,: niin tyttönikin juomahan opetan, \\
            ja me onnelliset oomme vaan. :,: \\
\hspace{10mm} \\
1.2 Servin Maijan mökki \\
Servin Maija \\
            :,: Servin Maijan mökki se seisoo \\
            korkealla mäellä. :,: \\
            Herrat sinne ajaa hevosella,  x 3 \\
            :,: jätkät saavat kävellä. :,: \\
\hspace{10mm} \\
            :,: Servin Maijan kakarat \\
            oli marakatin sukua. :,: \\
            Niit' oli puussa ja niit' oli maassa,  x 3 \\
            :,: eikä niillä ollunna lukua. :,: \\
\hspace{10mm} \\
            :,: Hindu ja bambu ja banaani \\
            ovat etelän hedelmiä juu. :,: \\
            Eikä niitä voida verratakaan,  x 3 \\
            :,: potaattiin ja hailiin juu. :,: \\
\hspace{10mm} \\
            :,: Servin Maija se kahvia keittää \\
            vanhassa saappaan varressa. :,: \\
            Kyllä sillä ompi pannukin,  x 3 \\
            :,: mutta se on “kanissa”. :,: \\
\hspace{10mm} \\
            :,: Servin Maija se sanoi, \\
            että mennään meille kotia. :,: \\
            Juodaanpa kuppi kahvia,  x 3 \\
            :,: ja pari lasii totia. :,: \\
\hspace{10mm} \\
            :,: Servin Maija nätti flikka \\
            kaadas ryyppy minulle. :,: \\
            Minähän se varmaan ensi vuonna  x 3 \\
            :,: oon sulhaspoika sinulle. :,: \\
\hspace{10mm} \\
1.3 Lapuan poika \\
Lapuan virran vesi \\
            Eikä se ollut siunattua, \\
            se Lapuan virran vesi juu, \\
            jolla pappi ensi kerran \\
            tämän pojan pesi juu. \\
\hspace{10mm} \\
            Piru itte vettä kantoi, \\
            virrasta verisellä sangolla. \\
            Kuolleen sällin sääriluulla \\
            hämmenteli sitä pankolla. \\
\hspace{10mm} \\
            Enkä mä ollut kuin viidentoista, \\
            kun haassa mä halasin lammasta. \\
            Pässi piru katteli airan raosta \\
            ja kiristeli mulle hammasta. \\
\hspace{10mm} \\
            Lisäsäkeistö: \\
\hspace{10mm} \\
            Enkä mä ollu kuin viidentoista, \\
            kun vaarini kirveellä lopetin. \\
            Siitäpä asti on kotinani ollu \\
            tuo Vaasan kuuluisa hotelli. \\
\hspace{10mm} \\
1.4 Tenttu-ukot hyppikää \\
Hei tenttu-ukot \\
            :,: Hei tenttu-ukot hyppikäätte, \\
            viel' on apteekit auki. :,: \\
            Vain hetken kestää humalaa, \\
            ja se on helkkarin mukavaa. \\
            Hei tenttu-ukot hyppikäätte, \\
            viel' on apteekit auki. \\
\hspace{10mm} \\
1.5 Rullaati rullaa \\
Murheisna miesnä \\
            Murheisna miesnä jos polkusi kuljet, \\
            konstin mä tiedän mi auttavi tuo: \\
            sä ennenkuin kuolossa silmäsi suljet, \\
            piiriimme istu ja laula ja juo! \\
            :,: Hei rullaati rullaati rullaati rullaa, \\
            rullaati rullaati rullallalei! :,: \\
\hspace{10mm} \\
            Elo on lyhyt kuin lapsella paita, \\
            muuta kai sanoa siitä en saa, \\
            lauluista murheilles karsina laita, \\
            veljet, on tässä meill' riemujen maa. \\
            :,: Hei rullaati rullaati rullaati rullaa, \\
            rullaati rullaati rullallalei! :,: \\
\hspace{10mm} \\
            Kevättä kestää vain neljännes vuotta, \\
            riennä siis joukkohon riemuitsevain. \\
            Ystävä, nuori et liene sä suotta, \\
            kätesi anna ja laulele ain'. \\
            :,: Hei rullaati rullaati rullaati rullaa, \\
            rullaati rullaati rullallalei! :,: \\
\hspace{10mm} \\
1.6 Surunpäiviä viettämään \\
Surun päiviä \\
            Ja se tyttö, joka rakkautens' kanssa kamppailee, \\
            se ompi niinkuin sammakko, mi suossa kurnailee. \\
            :,: Ei mutta surunpäiviä viettämään on tämä \\
            joukko vielä liian nuori. :,: \\
\hspace{10mm} \\
            Se tähti, joka taivahalta alas tipahtaa, \\
            se pyörii niinkuin pyssynluoti nurmikedolla. \\
            :,: Ei mutta surunpäiviä... \\
\hspace{10mm} \\
            Ja se poika, joka rakkautens' kanssa kamppailee, \\
            se ompi niinkuin orpo piru kuusen oksalla. \\
            :,: Ei mutta surunpäiviä... \\
\hspace{10mm} \\
1.7 Mustalaisen Veera \\
Raastuvan penkillä (painokset 1-11) \\
            Raastuvan penkillä halasin \\
            minä mustalaisen Veeraa. \\
            Raastuvan penkillä halasin \\
            minä mustalaisen Veera-ra-ra-ra, \\
            Veera se vapisi ja penkki se natisi \\
            ja orkar du inte mera-ra-ra-ra. \\
            Veera se vapisi ja penkki se natisi \\
            ja orkar du inte mera. \\
\hspace{10mm} \\
            På rådstugans bänken där halsade \\
            jag zigenarflickan Vera. \\
            På rådstugans bänken där halsade \\
            jag zigenarflickan Vera-ra-ra-ra, \\
            Vera hon darrade \\
            och bänken den knarrade \\
            och ekkös sä jaksa enää-nä-nä-nä. \\
            Vera hon darrade \\
            och bänken den knarrade \\
            och ekkös sä jaksa enää. \\
\hspace{10mm} \\
1.7 Ikkunalasin laulu \\
Raastuvan penkillä (painokset 12->) \\
            :,: Poika se likkojen kammarin klasista \\
            nauloja irki nyppi. :,: \\
            :,: Siitä on menty ennenkin, ja \\
            siitä mennään nytkin! :,: \\
\hspace{10mm} \\
1.8 Vahtimestari \\
Hei vahtimestari \\
            :,: Hei, vahtimestari! \\
            Tule, täytä lasini, \\
            vie tyhjät pullot pois! :,: \\
            :,: Kyllä enemmänkin juodaan, \\
            jos vain lisää tuodaan \\
            ja laskut maksetaan, jos jaksetaan. :,: \\
\hspace{10mm} \\
            :,: Hei, vahtimestari! \\
            Tule, täytä lasini, \\
            vie kuolleet fuksit pois! :,: \\
            :,: Ja jos mä kerran kuolen, \\
            kyll' kilta pitää huolen, \\
            ett' pokkamonttuun minut haudataan. :,: \\
\hspace{10mm} \\
            Lisäsäkeistö: \\
\hspace{10mm} \\
            :,: Hei, vahtimestari! \\
            Tule, täytä suoleni, \\
            se onkin minun paras puoleni! :,: \\
            :,: Ja kun mä kerran kuolen,  \\
            kyll’ kilta pitää suolen,  \\
            se vuosijuhlalahjaks annetaan. :,: \\
\hspace{10mm} \\
1.9 On elämämme vallan ihanaa \\
Ottakaa taas jo naukkukin \\
            Ottakaa taas jo naukkukin, \\
            ei tässä auta nukkua, \\
            kun kello lyöpi yksi, \\
            on aika jo kukkua. \\
            Jos meitä henttu haukkuukin, \\
            jos keppi seläs' paukkuukin, \\
            mies siitä viis veisaa, \\
            hän laulaa ja juo. \\
\hspace{10mm} \\
            On elämämme vallan ihanaa, \\
            kunhan vain juoda saa viinaa tilkkasen \\
            ja sitten mennään kotiin makaamaan, \\
            se on vallan paikallaan \\
            muijan vierehen. \\
\hspace{10mm} \\
            Olkohon hurtti huumori \\
            ja repäisevä meininki. \\
            Ei meitä surut ruoki, \\
            se selvä on tietenkin, \\
            ottakaa lasi pohjaan vaan, \\
            täyttyyhän se taas aikanaan, \\
            kernaasti kyyppi tuopi, \\
            kun laulaa ja juo. \\
\hspace{10mm} \\
            On elämämme... \\
\hspace{10mm} \\
1.10 Oi jos takaisin saisin \\
Oi jos mä takaisin \\
            Oi, jos mä takaisin saisin sen ajan, \\
            kun kapakka mulle oli tuntematon, \\
            :,: se viini, joka lasissa helmeili kerran, \\
            se juotuna turmion tuottanut on. :,: \\
\hspace{10mm} \\
            Ennen minä lauloin kuin lintunen pieni, \\
            joka pesästään maailmalle lentänyt on. \\
            :,: En tiennyt minä silloin, että maailma milloin, \\
            se minutkin pauloihinsa ottava on. :,: \\
\hspace{10mm} \\
            Onneni tähden ne taivahalla loisti, \\
            en suruja silloin minä tuntenutkaan. \\
            :,: Kun ihana impi, se minua lempi, \\
            sanoi: “emmehän erkane milloinkaan.” :,: \\
\hspace{10mm} \\
            Tuomi se huojui ja kukkaset tuoksui, \\
            kun tyttöni mun rinnallain asteli. \\
            :,: “Sinä armaani olet”, näin tyttöni lausui \\
            ja suudelman painoi mun huulilleni. :,: \\
\hspace{10mm} \\
            Ystäviä mulla oli maailmassa monta, \\
            jotka kapakan riemuista nauttivat vaan. \\
            :,: Ja sanoivat mulle, et sä lemmestä tiedä, \\
            kun kapakka sulle on tuntematon. :,: \\
\hspace{10mm} \\
            Ja he sanoivat mulle, oi mukahan tule \\
            kyllä riemuja sullekin riittävä on. \\
            :,: Siellä musiikin soidessa huolet ne haihtuu, \\
            ja maljat ne surusi poistava on. :,: \\
\hspace{10mm} \\
            Nyt istun mä kapakassa ystävien kanssa \\
            ja kaikki ne juovat minun onnekseni, \\
            :,: ja he iloitsevat vielä, että kuljen minä tiellä, \\
            joka johtava on minut turmiohon. :,: \\
\hspace{10mm} \\
            Onneni tähdet ne taivaalla sammui, \\
            kun juomarin tielle mä joutunut oon. \\
            :,: Sinä tyttöni kallis, mulle anteeksi anna, \\
            vaikka juomari olen minä auttamaton. :,: \\
\hspace{10mm} \\
1.11 Merimies \\
Kun myrsky käy \\
            :,: Kun myrsky käy \\
            ja ukkonen jyllää \\
            hei ruma sana juu. :,: \\
            :,: niin pikipaita \\
            se riemuin \\
            no-no-no-no-nostaa riemuin \\
            ko-ko-ko-ko-koko purjehen. :,: \\
\hspace{10mm} \\
            :,: Käy merimies maihin \\
            heijuma reijuma \\
            hei ruma sana juu. :,: \\
            :,: siit' tytöt tykkää \\
            ja antaa \\
            kuti-kuti-kuttaa antaa \\
            puti-puti-puttaa pusuja. :,: \\
\hspace{10mm} \\
1.12 Juomaripoika \\
Pappani pellon perillä \\
            Juomaripojan hevonen se seisoo \\
            joka kapakalla, joka kapakalla \\
            Juomaripojan hevonen se seisoo \\
            joka kapakalla. \\
\hspace{10mm} \\
            Mitä se tekee juomaripoika \\
            heilal' napakalla, heilal' napakalla \\
            Mitä se tekee juomaripoika \\
            heilal' napakalla? \\
\hspace{10mm} \\
            Juomaripojan hevonen se seisoo \\
            aina ilman heinii, aina ilman heinii. \\
            Ja juomari itse kapakassa \\
            silittelee seinii. \\
\hspace{10mm} \\
            Juomaripojan hevonen se seisoo \\
            aina ilman lointa, aina ilman lointa. \\
            Ja juomaripoika se juomisen jälkeen \\
            on aina ilman tointa. \\
\hspace{10mm} \\
            Mitä se tekee se juomaripoika \\
            taloss' emännällä, taloss' emännällä? \\
            Kun ottaa piian ja maksaa palkan \\
            niin pääsee vähemmällä. \\
\hspace{10mm} \\
            Siitähän sen aina juomarin tuntee, \\
            kun se aina laulaa, kun se aina laulaa. \\
            Ja juomaripoika se kätensä kietoo \\
            joka tytön kaulaan. \\
\hspace{10mm} \\
1.13 Maailmanmarkkinoilla \\
Maailman Matti \\
            Ja täss' on kans Matti tässä maailmassa, \\
            vaikka varreltansa matala. \\
            Eikä ilojani suruiksi muuttaa voi \\
            tämä maailma katala. \\
            En mä luotu ole itkemään, \\
            enkä ohdakkeita kitkemään. \\
            :,: Hurraa, hurraa hoi, \\
            minun lauluni soi \\
            näillä maailmanmarkkinoilla. :,: \\
\hspace{10mm} \\
            Näitä markkinoita maailmassa matkaillut \\
            olen vuosia monia. \\
            Enkä riiaamatta ole minä jättänyt \\
            noita tyttöjä somia, \\
            mutta miniätä mammallein, \\
            sitä tuonut vielä mä en. \\
            :,: Hurraa, hurraa... \\
\hspace{10mm} \\
            Olen pienestä pojasta osannut \\
            lyödä kortilla nakkia, \\
            vaikka pappa ja mamma toivoivat \\
            tästä pojasta pappia. \\
            Sitä mammani suri hei, \\
            kun pappia tullut ei. \\
            :,: Hurraa, hurraa... \\
\hspace{10mm} \\
            Ja vaikka orjantappurat ja ohdakkeet \\
            minun polkuni peittäisi, \\
            niin laulamasta en minä lakkaisi \\
            enkä ilojani heittäisi. \\
            En huoli minä suruja, \\
            etsin onnen muruja. \\
            :,: Hurraa, hurraa... \\
