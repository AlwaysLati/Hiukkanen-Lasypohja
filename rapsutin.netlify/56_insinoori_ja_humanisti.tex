\song{Insinööri ja humanisti (\textit{56})}{Päivänsäde ja menninkäinen}

            Assari kun päätti tentin \\
            siskoistaan jäi jälkeen sentin \\
            humanisti viimeinen. \\
            Hämärä jo Manseen hiipi, \\
            humanisti ulos kiiti \\
            juuri aikoi nousta linjuriin. \\
            Kun insinöörin fiksun näki vastaan \\
            tulevan, se juuri oli noussut nyssestään. \\
            Kas insinööri ennen päivänlaskua ei voi \\
            milloinkaan lähtee lafkaltaan. \\
\hspace{10mm} \\
            Katselivat toisiansa, \\
            insinööri rinnassansa \\
            tunsi kummaa leiskuntaa. \\
            Sanoi: “Poltat Marlboroa, \\
            mutt’ en ole eläissäni \\
            nähnyt mitään yhtä ihanaa! \\
            Ei haittaa vaikka röökisi mun yskiväksi saa, \\
            on astmaisena hyvä asustaa. \\
            Käy kanssani niin Herwantaan mä näytän sulle tien \\
            ja sinut Mikontaloon vien!” \\
\hspace{10mm} \\
            Nisti vastas: “Nööri kulta, \\
            teekkarit vie järjen multa \\
            enkä toivo hourulaan. \\
            Pois mun täytyy heti mennä, \\
            ellen kohta kotiin ennä \\
            niin en hetkeäkään lukee saa!” \\
            Niin lähti kaunis humanisti, mutta vieläkin, \\
            kun insinööri öisin integroi, \\
            hän miettii miksi toinen täällä lukutoukka on \\
            ja toinen nablaa rakastaa. \\
