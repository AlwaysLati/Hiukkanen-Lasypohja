\song{Rosvo-Roope (\textit{617})}{}

        Jos täytätte mun lasini, niin tahdon kertoa \\
        surullisen tarinan, jol' ei oo vertoa. \\
        Se on laulu merirosvosta Roope nimeltään, \\
        hän sydämiä särki missä joutui kiertämään. \\
\hspace{10mm} \\
        Hän kaunis oli kasvoiltaan ja nuori iältään, \\
        ja opetuksen saanut oli omalta isältään. \\
        Vaan tyttö jota lempi, hän oli Patteri, \\
        ja siksi tuli Roopesta nyt julma ryöväri. \\
\hspace{10mm} \\
        Viel' Itämeren rannikot ne Roopen muistavat \\
        ja naiset Pietarinkin vielä päätään puistavat \\
        ja Suomenlahden kaupungit tuns' Roope-ryövärin \\
        ja suuri oli pelko vielä Kokkolassakin. \\
\hspace{10mm} \\
        Ei neito Viron rannikon unhoita milloinkaan, \\
        kun Rosvo-Roopen kanssa hän joutui kapakkaan. \\
        He söivät mitä saivat ja joivat tuutingin, \\
        ja tyttö poltti sydämens', mut' Roope – sikarin. \\
\hspace{10mm} \\
        Ei tiennyt impi Oolannin, kuink' oli laitansa, \\
        kun Rosvo-Roope pestäväks' vei hälle paitansa. \\
        Yks' nappi oli irti, hän ompeli sen kiin', \\
        ja samalla hän ompeli myös sydämensä siin'. \\
\hspace{10mm} \\
        Sai Roope viimein palkkansa, hän on nyt Suomessa \\
        jossain jokivarressa lie lossivahtina. \\
        Hän lesken eessä nöyrtyi ja joutui naimisiin, \\
        ja sillä lailla Rosvo-Roope hiljaa hirtettiin. \\
\hspace{10mm} \\
        :,: Oi Roope, oi Roope, \\
        mikset pitänyt sä nappejasi kii-ii-i. :,: \\
