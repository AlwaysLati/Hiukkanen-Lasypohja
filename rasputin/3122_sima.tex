\song{Sima (\textit{3122})}{Sika (Vompatti / Perttu Ruuska)}

Wapunviettomme alkaa jo maaliskuussa \\
Panomies silloin on päävastuussa \\
Hält' sitruunan pilkonta luonnistuu \\
Ja sitten kun koittaa huhtikuu \\
Niin fuksit laulaa: rusinakin on ehdoton \\
\hspace{10mm} \\
Sima – nurkassa käydä saa \\
Sima – sen setä pullottaa \\
Sima – ja setä litran juo \\
Sima – se wappumielen tuo \\
\hspace{10mm} \\
Minä tonkassa kuplivaa litkua katsoin \\
Oottavin ilmein, pyytävin vatsoin \\
Wappuna simaani vaihtais en \\
vaik jallua tarjottais litranen \\
Ja fuksit laulaa: rusinakin on ehdoton \\
\hspace{10mm} \\
Sima – käymistä ootetaan \\
Sima – yhdessä kuolataan \\
Sima – ja täti mielellään \\
Sima – maistelee kielellään \\
\hspace{10mm} \\
Vapputorilla ihmiset sokkona ryntää \\
Teekkari puistoa naamallaan kyntää \\
Sitä ihmiset katsoo kateissaan \\
Kun simapullo näkyy taskussaan \\
Ja fuksit laulaa: rusinakin on ehdoton \\
\hspace{10mm} \\
Sima – sitruuna haistetaan \\
Sima – nyt sitä maistetaan \\
Sima – on käynyt kunnolla \\
Sima – on hiivat pohjalla \\
\hspace{10mm} \\
Kato setä on laittanut käymään kyllä \\
Rusinakerros on simassa yllä \\
Mausta myös hunajan erottaa \\
Kun maksaansa simalla verottaa \\
Fuksit laulaa: rusinakin on ehdoton \\
\hspace{10mm} \\
Sima – lopulta joimme sen \\
Sima – herkkujuoman sitrushappoisen \\
Sima – nyt kebun setä syö \\
Sima – taas pullon pöytään lyö \\
\hspace{10mm} \\
Wappu täynnä on Gambinaa, olutta,\\
skumppaa, \\
Jallua, ferraa ja tietenkin votkaa \\
Kun loppuu aatto ja alkaa yö \\
Ihminen rusinat simasta syö \\
\hspace{10mm} \\
Sima – lalllaallaaa-aa \\
Sima – trallallaa-aa \\
Sima – trallallaa-aa \\
Sima – trallaallaaalla-aa \\